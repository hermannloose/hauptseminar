\documentclass[11pt]{article}
\usepackage{fontspec}
\usepackage{ngerman}

\setromanfont{Charis SIL}

\title{Control-Flow Integrity Checking}
\author{Hermann Loose}

\begin{document}

\maketitle

\section{Einführung}

% Doofer Satzanfang.
Ungültiger Kontrollfluss in Programmen stellt ein Problem sowohl für die
Sicherheit der Rechnersysteme auf denen diese Programme laufen als auch für die
Verlässlichkeit der von ihnen ausgeführten Berechnungen dar.

% Blah blah?
Es werden in der Folge zwei Verfahren betrachtet, die sich jeweils einem dieser
beiden Aspekte widmen und einander dabei in der Vorgehensweise zum Teil sehr
ähnlich sind.

\section{Control-Flow Integrity}

Das in \cite{abadi-2005-control-msr} vorgestellte Verfahren \emph{Control-Flow
Integrity} (CFI) zielt darauf ab, Angriffe zu verhindern, die auf dem Verlassen des
erlaubten Kontrollflusses in einem Programm aufbauen.

Die Autoren argumentieren, dass viele der bisher gegen solche Angriffe
eingesetzten Mechanismen in ihrer Wirksamkeit beschränkt seien, da ihnen ein
realistisches Angreifermodell fehlen würde und sie sich selten auf formale
Schlüsse und häufig auf verborgene Annahmen stützten.

Die Antwort auf diese Probleme seien einfach verständliche Verfahren mit
dennoch umfassenden Garantien gegenüber starken Angreifern. Zudem sollten
praxistaugliche Techniken möglichst auf existierenden Code bzw. sogar auf
existierende Binaries anwendbar sein.

CFI als eine mögliche Lösung wurde von den Autoren für Windows auf der x86
Architektur entwickelt und getestet.

\subsection{Grundlagen}

% TODO(hermannloose): An welcher Stelle Theorie einbauen?

% FIXME(hermannloose): Artet evtl. in Kopie aus.
\subsection{Instrumentierung}

% FIXME(hermannloose): Artet evtl. in Kopie aus.
\subsection{Annahmen}

\subsection{Messungen}

Die in \cite{abadi-2009-control-tissec} diskutierten Messungen beziehen auf
bekannte Benchmarks aus dem SPEC-Paket. Sie wurden auf einem Pentium 4 x86 mit
1.8GHz und 512MB RAM unter Verwendung von Windows XP SP2 im \emph{Safe Mode},
bei dem ein Großteil der Dienste und Kernelmodule deaktiviert ist,
durchgeführt. Die untersuchten Programme wurden in Microsoft Visual C++ 7.1 mit
allen verfügbaren Optimierungen compiliert.

Die ermittelten Werte stellen jeweils den Durchschnitt von drei Durchläufen
dar, die Standardabweichung von weniger als einem Prozent wurde von den Autoren
als vernachlässigbar angesehen.

% Kritik: wenig sicherheitsrelevante Ergebnisse, bereits 2004 im ersten Entwurf
% nur Sektion für Performancemessungen angelegt!

% FIXME(hermannloose): Besseren Titel finden.
\subsection{CFI als Grundlage für\\andere Verfahren}

% TODO(hermannloose): Zitate für IRMs und SMAC?
Aufgrund seiner Garantien für den Kontrollfluss von Programmen kann CFI zudem
als Grundlage für den Einsatz bzw. die Optimierung und Vereinfachung einer
Reihe anderer, ebenfalls softwarebasierter Mechanismen dienen. Die in
\cite{abadi-2009-control-tissec} beleuchteten sind \emph{Inlined Reference
Monitors} (IRMs) und \emph{Software Memory Access-Control} (SMAC).

% wie erkläre ich die zwei Verfahren, ohne hier einfach das Paper übersetzt zu
% reproduzieren? (oder ist das legitim?)

% FIXME(hermannloose): Anderen Titel finden.
% Ausblick nach sieben Jahren ist etwas müßig.
\subsection{Ausblick}

% Kritik: an dem Thema hat sich seit 2005 nichts getan
\cite{abadi-2009-control-tissec} erschien zuerst im Februar 2005 als Technical
Report von Microsoft Research \cite{abadi-2005-control-msr}. Spätere
Veröffentlichungen 2005 und 2007 sind inhaltlich identisch. Zum Problem des
Umgangs mit selbstmodifizierendem Code und Codegenerierung zur Laufzeit erfolgt
in \cite{abadi-2005-control-msr} der Verweis „we are working on […] handling
runtime code generation and other dynamic additions of code“, seit 2007 und in
\cite{abadi-2009-control-tissec} lautet diese Stelle nunmehr „we have
considered working on […] handling runtime code generation and other dynamic
additions of code“, was nahelegt, dass hier kein weiterer Fortschritt erfolgt
ist. % TODO(hermannloose): Zitate checken.

% beleuchten: ist Overhead ein Problem?

\section{Control-Flow Checking by\\Software Signatures}

% FIXME(hermannloose): Dopplung des Titels.
\subsection{Einführung}

% Ansatz zielt auf Fehlertoleranz ab

\subsection{Konzept}

\subsection{Simulationsergebnisse}

% TODO(hermannloose): Gab es hier weitere Paper?

\pagebreak
%\section{Quellen}
\bibliographystyle{unsrt}
\bibliography{paper}

\end{document}
