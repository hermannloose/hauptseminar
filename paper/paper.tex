\documentclass[10pt,twocolumn]{article}
\usepackage{fontspec}
\usepackage{ngerman}
\usepackage[top=2.5cm,left=2cm,right=2cm,bottom=2.5cm]{geometry}

\setromanfont{Charis SIL}

\title{Control-Flow Integrity Checking}
\author{Hermann Loose}

\begin{document}

\maketitle

\section{Einführung}

% ungültiger Kontrollfluss stellt sowohl ein Problem für die Sicherheit von
% Systemen als auch für die Korrektheit der von ihnen ausgeführten Berechnungen
% dar

\section{Control-Flow Integrity}

% FIXME(hermannloose): Dopplung des Titels.
\subsection{Einführung}

% Ansatz zielt auf Sicherheit ab

% TODO(hermannloose): An welcher Stelle Theorie einbauen?

% FIXME(hermannloose): Artet evtl. in Kopie aus.
\subsection{Instrumentierung}

% FIXME(hermannloose): Artet evtl. in Kopie aus.
\subsection{Annahmen}

% FIXME(hermannloose): Titel finden.
\subsection{CFI als Grundlage für\\andere Verfahren}

\subsection{Messungen}

% Kritik: wenig sicherheitsrelevante Ergebnisse

\subsection{Ausblick}

% Kritik: an dem Thema hat sich seit 2005 nichts getan

% beleuchten: ist Overhead ein Problem?

\section{Control-Flow Checking by\\Software Signatures}

% FIXME(hermannloose): Dopplung des Titels.
\subsection{Einführung}

% Ansatz zielt auf Fehlertoleranz ab

\subsection{Konzept}

%\section{Quellen}
\bibliographystyle{plain}
\bibliography{paper}

\end{document}
