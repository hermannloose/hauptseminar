\documentclass[serif,slidestop,compress,red]{beamer}
%\usepackage[utf8]{inputenc}

\usetheme{Antibes}

\usepackage{fontspec}
%\fontspec{Ubuntu}
%\fontspec{Helsinki}
%\fontspec{Vollkorn}
\usepackage{xltxtra}
\setmainfont{Ubuntu}
\setmonofont{Courier 10 Pitch}
%\setromanfont{Helsinki}

\beamertemplatenavigationsymbolsempty

\title{Compiler-Unterstützung für \\ Control-Flow Integrity Checking}
\subtitle{Vortrag im Hauptseminar Betriebssysteme}
\author{Hermann Loose}

\begin{document}

\frame[plain]{\titlepage}

\section{Control-Flow Integrity}

\subsection{Überblick}

\begin{frame}
  \frametitle{Angreifermodell}
  \begin{itemize}
    \item volle Kontrolle über Datensegment
    \item mögliche Kontrolle über Modul oder Thread im selben Adressraum
  \end{itemize}
\end{frame}

\begin{frame}
  \frametitle{Anwendbarkeit für Angriffsszenarien}
  \begin{itemize}
    \item klassische, Stack-basierte Pufferüberläufe
    \item Heap-basierte jump-to-\texttt{libc} Angriffe
  \end{itemize}
  Für ein vereinfachtes Maschinenmodell existiert ein formaler Beweis in [2].
\end{frame}

\begin{frame}
  \frametitle{Control-Flow Graph (CFG)}
  \begin{itemize}
    \item bestimmt erlaubte Ausführungsreihenfolgen
    \item muss im Vorhinein feststehen:
    \begin{itemize}
      \item Analyse von Quellcode
      \item Analyse von Binaries
      \item Profiling der Ausführung
      \item als Ausdruck einer vorgegebenen Security Policy
    \end{itemize}
    \item hier: statische Analyse von Binaries
  \end{itemize}
\end{frame}

\begin{frame}
  \frametitle{Annahmen}
  \begin{itemize}
    \item[UNQ] Unique IDs
    \begin{itemize}
      \item IDs kommen nicht zufällig im Codesegment vor
      \item Größe des Suchraumes, z.B. 32 Bit
    \end{itemize}
    \item[NWC] Non-Writable Code Memory
    \begin{itemize}
      \item auf den meisten Systemen gegeben
    \end{itemize}
    \item[NXD] Non-Executable Data Memory
    \begin{itemize}
      \item in Hardware auf x86
      \item in Software möglich\footnote{PaX — http://pax.grsecurity.net}
    \end{itemize}
  \end{itemize}
\end{frame}

\begin{frame}
  \frametitle{Instrumentierung}
  \begin{itemize}
    \item entsprechend dem CFG
    \item umfasst Quell- und alle möglichen Zielinstruktionen von Sprüngen mit berechnetem Ziel
    \item vor jedem Ziel eine \texttt{label}-Instruktion mit der ID der Äquivalenzklasse des Ziels
  \end{itemize}
\end{frame}

\subsection{Implementierung}

\subsection{CFI als Basis für andere Mechanismen}

\begin{frame}
  \frametitle{CFI als Grundlage für IRMs}
\end{frame}

\begin{frame}
  \frametitle{SMAC: Generalized SFI}
\end{frame}

\section{Software Signatures}

\section{}

\begin{frame}
  \frametitle{Quellen}
  \begin{enumerate}
    \item \emph{Control-Flow Integrity} \\ M. Abadi, M. Budiu, Ú. Erlingsson, J. Ligatti — 2005
    \item \emph{A Theory of Secure Control Flow} \\ M. Abadi, M. Budiu, Ú. Erlingsson, J. Ligatti — 2005
    \item \emph{Control-Flow Checking by Software Signatures} \\ N. Oh, P. Shirvani, E. McCluskey — 2000
  \end{enumerate}
\end{frame}

\end{document}
